%% -*- TeX-master: t -*-

\documentclass{acm_proc_article-sp}
%% \documentclass{sig-alternate}


%%%%%%%%%%%%%%%%%%%%%%%%%%%%%%%%%%%%%%%%%%%%%%%%%%%%%%%%%%%%%%%%%%%%%%%%
%%
%% standard texmf packages
%%

\usepackage{nicefrac}
\usepackage{fancyvrb}
\usepackage{xspace}

\usepackage[bookmarks,pdfauthor={Ben Liblit, Mayur Naik, Alice X.
  Zheng, Alex Aiken, and Michael I.
  Jordan},pdfpagemode=UseOutlines]{hyperref}


%%%%%%%%%%%%%%%%%%%%%%%%%%%%%%%%%%%%%%%%%%%%%%%%%%%%%%%%%%%%%%%%%%%%%%%%
%%
%% unique to this paper
%%

\usepackage{Autoref}

%% assorted handy macros
\newcommand{\moss}{\textsc{Moss}\xspace}
\newcommand{\termdef}[1]{\textit{#1}}
\newcommand{\prob}{\mbox{\textit{Prob}}}
\newcommand{\fail}{\mbox{\textit{Fail}}}
\newcommand{\crash}{\mbox{\textit{Crash}}}
\newcommand{\context}{\mbox{\textit{Context}}}
\newcommand{\increase}{\mbox{\textit{Increase}}}


%%%%%%%%%%%%%%%%%%%%%%%%%%%%%%%%%%%%%%%%%%%%%%%%%%%%%%%%%%%%%%%%%%%%%%%%
%%
%% remove the following before submission!
%%

\usepackage{color}
\newcommand{\placeholder}[1]{{\color[cmyk]{0,0.61,0.87,0}[#1]}}
\pagenumbering{arabic}


%%%%%%%%%%%%%%%%%%%%%%%%%%%%%%%%%%%%%%%%%%%%%%%%%%%%%%%%%%%%%%%%%%%%%%%%
%%
%% front matter
%%

\title{Statistical Debugging in the Presence of Multiple Bugs
  %%
  \thanks{\placeholder{This grant boilerplate is from the PLDI'03
      paper.  Newer text is needed here.}  This research was supported
    in part by NASA Grant No.\ NAG2-1210; NSF Grant Nos.\ EIA-9802069,
    CCR-0085949, ACI-9619020, and IIS-9988642; DOE Prime Contract No.\ 
    W-7405-ENG-48 through Memorandum Agreement No.\ B504962 with LLNL;
    and a Lucent GRPW Fellowship.  The information presented here does
    not necessarily reflect the position or the policy of the
    Government and no official endorsement should be inferred.}}

\numberofauthors{3}

\makeatletter
\newcommand*{\eecsMark}[0]{\@fnsymbol{2}}
\newcommand*{\statMark}[0]{\@fnsymbol{3}}
\newcommand*{\stanMark}[0]{\@fnsymbol{4}}
\makeatother
\newcommand*{\eecs}[0]{\textsuperscript{\eecsMark}}
\newcommand*{\stat}[0]{\textsuperscript{\statMark}}
\newcommand*{\both}[0]{\textsuperscript{\eecsMark, \statMark}}
\newcommand*{\stan}[0]{\textsuperscript{\stanMark}}

\newcommand{\moreauthors}[0]{\end{tabular}\\\vspace{-.5\baselineskip}\begin{tabular}{c}}

\author{
  \alignauthor Ben Liblit \eecs \\
  \alignauthor Mayur Naik \stan \\
  \alignauthor Alice X.\ Zheng \eecs \\
  \moreauthors
  \global\multiply\auwidth by 3
  \global\divide\auwidth by 2
  \alignauthor Alex Aiken \stan \\
  \alignauthor Michael I.\ Jordan \both
  \moreauthors
  \alignauthor
  \affaddr{\eecs Department of Electrical \\ Engineering and Computer Science} \\
  \affaddr{\stat Department of Statistics} \\
  \affaddr{University of California, Berkeley} \\
  \affaddr{Berkeley, CA 94720-1776}
  \alignauthor
  \affaddr{\stan Computer Science Department} \\
  \affaddr{353 Serra Mall} \\
  \affaddr{Stanford University} \\
  \affaddr{Stanford CA 94305-9025}
}

\bibliographystyle{abbrv}


%%%%%%%%%%%%%%%%%%%%%%%%%%%%%%%%%%%%%%%%%%%%%%%%%%%%%%%%%%%%%%%%%%%%%%%%
%%
%%  document body
%%

\begin{document}

\conferenceinfo{PLDI'04,}{June 9--11, 2004, Washington, DC, USA.}
\CopyrightYear{2004}
%% \crdata{}
\maketitle

\begin{abstract}
\placeholder{Abstract needed here.}
\end{abstract}

\category{D.2.5}{Software Engineering}{Testing and
  Debugging}[distributed debugging]
%%
\category{G.3}{Mathematics of Computing}{Probability and
  Statistics}[correlation and regression analysis]
%%
\category{I.5.2}{Pattern Recognition}{Design Methodology}[feature
  evaluation and selection]

\terms{Experimentation, Performance, Reliability}

\keywords{bug isolation, random sampling, assertions, feature
  selection, statistical debugging, logistic regression}

\placeholder{The preceding categories, terms, and keywords were copied
  from the PLDI'03 paper.  Some may need to be changed.  Note that the
  categories and terms are \emph{not} free-form; see
  \url{http://www.acm.org/sigs/pubs/proceed/sigguide-v2.1sp.htm}
  sections 2.3.2 and 2.3.3 for details.}


\section{Introduction}
\label{sec:introduction}

\placeholder{Page budget: 1.}


\section{Background}
\label{sec:background}

\placeholder{Page budget: 1.}

We begin by reviewing the sampled instrumentation infrastructure.  We
highlight the ways in which the design constraints discussed earlier
\placeholder{reword if we don't do that in the introduction} impact
the kind and form of data collected.  The challenge taken up in
\Autoref{sec:algorithm} is to use this data in a way that helps
isolate problems in buggy software.

Software errors arise from unexpected interactions between inputs,
program internal state, and dynamic decisions made as the program
runs.  Software engineers reason about programs at the level of source
code statements and variables, and this is the level at which program
monitoring can expose properties of interest.  However, complete
source-level tracing of program behavior is impractical for field
deployment.  No end user would accept the performance overhead or
network bandwidth required to collect and transmit such a trace.

Instead, we use a combination of client-side summarization and fair
random sampling.  The former controls limits storage and transmission
requirements while the later controls performance overhead.  However,
each also adds noise and uncertainty to the resulting data.  We
briefly review these two techniques and their implications here;
additional details including performance assessments have been
published elsewhere \cite{PLDI`03*141}.  Although our approach is
applicable to a wide variety of programming languages, our current
implementation and all examples that follow are targeted at C.

\subsection{Schemes and Summarization}

An \termdef{instrumentation scheme} is a set of rules describing what
information to collect at each program point of interest.  Our
instrumentation schemes are quite general but are selected to capture
behaviors which are likely to be of interest when hunting for bugs.
At present our system offers the following instrumentation schemes:
  
\begin{description}
\item[branches] Control flow is interesting.  Record which branch is
  taken at each conditional (equivalently, whether the conditional
  predicate was true or false).  The observation is made just after
  the predicate is evaluated but before the selected branch is taken.
  This scheme also applies to implicit conditionals in loops and
  logical operators (\texttt{\&\&}, \texttt{||}, \texttt{?:}).  Each
  branch induces one instrumentation site.
  
\item[returns] Function return values are interesting.  In C, the sign
  of a result often used to encode the success or failure of some
  operation.  At each scalar-returning function call site, record
  whether the returned value is negative, zero, or positive.  For
  pointer-returning calls, this implicitly reduces to \texttt{NULL} or
  non-\texttt{NULL}.  The observation is made just after the function
  returns but before the result is used by the original program.  An
  instrumentation site is added even if the source program discarded
  the return value, as ignoring returned error codes is a common
  source of bugs..  Each call induces one instrumentation site.
  
\item[scalar-pairs] Values of variables are interesting.  Many bugs
  concern boundary issues exposed by the relationship between a pair
  of variables, or between a variable and some program constant.  At
  each scalar assignment \texttt{x = \dots}, identify each
  \emph{other} same-typed in-scope variable $\mathtt{y}_i$ and each
  constant expression $\mathtt{c}_j$.  Record how often the new value
  for \texttt{x} is less than, equal to, or greater than each
  $\mathtt{y}_i$ and each $\mathtt{c}_j$.  The observation is made
  after both sides of the assignment have been evaluated but just
  before the assignment itself takes place.  This lets us compare
  \texttt{x} to \texttt{x} as well, effectively comparing the new and
  old values of the left-hand side.  Each compared-to $\mathtt{y}_i$
  or $\mathtt{c}_j$ is treated as a distinct instrumentation site;
  thus a single assignment may induce a large number of sites.
\end{description}

These schemes are quite broad; they represent a large set of wild
guesses as to what behavior may be interesting for any particular bug.
Engineers may enable or disable any mix of schemes in a single binary,
and may include or exclude entire regions of code on a per-file or
per-function basis.  For the most part, though, instrumenting an
executable simply requires switching to our instrumenting compiler.
By design the instrumentor requires minimal human intervention at
application build time.

In the returns and scalar-pairs schemes, each site takes a large space
of possible observations (e.g.\ the exact value returned by a call)
and reduces it to a much smaller group of equivalence classes (e.g.\ 
the sign of the returned value).  This is already a summarization of
behavior.  However, we must shrink the program profile even further,
down to a flat behavior profile whose size is both moderate and fixed
regardless of how long a given application runs.  Therefore, we do not
emit a stream of observations from each site.  Rather, each site
maintains small group of counters which tally how often each behavior
is observed.  Thus, each branch site has two counters: how often the
program branched true and how often it branched false.  Each returns
site uses three counters to record how often a negative, zero, or
positive value is seen.  Each scalar-pairs site uses three counters as
well to record how often the assigned value is less than, equal to, or
greater than the compared-to variable or constant.

These counters are maintained within the program's global address
space during execution.  Note that each site can be treated as a set
of predicates which form a complete, non-overlapping partition of the
behaviors observed by that site.  For example, at each scalar-pairs
comparison between \texttt{x} and \texttt{y}, one and only one of
$\mathtt{x} < \mathtt{y}$, $\mathtt{x} = \mathtt{y}$, or $\mathtt{x} >
\mathtt{y}$ can be true.  Thus one observation at one site always
updates exactly one of that site's counters.  An observation that
$\mathtt{x} < \mathtt{y}$ is therefore equivalent to an observation
that $\mathtt{x} \ge \mathtt{y}$.  This property will be used in
\Autoref{sec:algorithm} to synthesize new predicates out of existing
observations.

The post-execution feedback report is a dump of the counter values for
each instrumentation site.  Reducing a trace to a set of counters
prevents us from reasoning about relative time ordering of events
during execution.  However, it also means that program actions early
in execution remain just as visible as those much later.  This is in
stark contrast with traditional postmortem debugging tools which
expose only the final state of the program at the point of failure.

\subsection{Sparse Random Sampling}

Each individual instrumentation site is fast, but in large numbers
they will harm performance.  Controlling this overhead requires that
we skip some observations.  If sampling is sparse, then the common
case of skipping a sample can be optimized as a ``fast path'' For such
partial instrumentation to be statistically meaningful, it must be
fair in a strict sense: each site when crossed at run time must have
an identical and independent chance of being sampled or omitted
regardless of how that same decision was made for any other dynamic
crossing of any other site.

The requirement suggests tossing a biased coin at each site.  However,
generating these biased random bits would be slower than simply
performing all observations unconditionally.  We must amortize the
decision cost across larger numbers of sites.  Consider a sequence of
biased random bits with 0 (skip) much more common than 1 (sample).
Instead of storing the sequence of bits directly, record only how many
0's appear before the next 1: how many sites are to be skipped before
the next sample is taken.  This is the \termdef{inter-arrival time} of
1's, and can be computed directly by selecting random numbers from a
geometric distribution.  A geometric distribution with mean 100 gives
inter-arrival times for a stream where samples are taken on average
once per hundred opportunities.

Instrumented code uses a global ``next sample'' countdown to predict
upcoming observations.  At each site, we decrement the countdown by
one.  If that drives the countdown to zero, then we take an
observation and reset the countdown to a new geometrically distributed
random value.  This counter has a useful near-term predictive quality:
if the counter is 58, then we know that none of the next 58 sites
crossed will be sampled.  This fact lets us amortize the sample/skip
decision across larger regions of code.  Consider any program point at
the top of an acyclic region.  There are only a finite number of
forward paths within that region, and therefore a finite maximum
number of sites along those paths.  We may know, for example, that no
forward path through a given region crosses more than 24 sites.  If
the next sample is more than 24 sites away, then we know in advance
that no sample will be taken on this pass through the region.  We
instead branch off into an alternate copy of the code which contains
no instrumentation beyond simply updating the global next sample
countdown.  This maximum site count becomes the \termdef{threshold}
for that region.

\placeholder{\dots}

This is an explicit trade-off between overhead and noise.  Sparser
sampling yields lower overhead but more noise (uncertainty) in each
feedback report.  We compensate for this uncertainty by looking for
broad trends across many runs.  With thousands or millions of users,
any individual feedback report may be noisy but the behavior of the
system as a whole will converge on an accurate representation of real
program behavior.

\section{Algorithm}
\label{sec:algorithm}

\placeholder{ We need to make sure we mention somewhere that (1) predicates are conceptually sampled after the line
is executed and (2) we transform our counters into real predicates before running this algorithm.}

This section presents the algorithm that we have developed for
isolating bugs in programs where multiple bugs are present
simultaneously.  As discussed in \Autoref{sec:background}, our
approach is to count the number of times we observe pre-specified
predicates at each program program point to be true during program
execution.  Because our system has no \textit{a priori} knowledge of
what bugs may be in the program, or even any model of what the program
does, our strategy is to make the set of predicates large in the
expectation that if the predicate set covers enough facets of the
program, every bug will be correlated with some predicate in the set.

In fact, our instrumentation strategies generate very large sets
of predicates; in a typical application tens of thousands of distinct
predicates are randomly sampled during program execution.  Because the
number of distinct bugs in a program is (hopefully!) orders of
magnitude smaller than the number of instrumented predicates, the
algorithmic problem is at least as much about discarding irrelevant
predicates as it is about identifying relevant predicates.  This
observation is reflected in our algorithm, which consists of two phases:
\begin{enumerate}
\item Eliminate predicates that are not predictive of program failure.

\item Rank the predicates that remain.  The higher a predicate is ranked,
the more confident we are that is involved in a bug.
\end{enumerate}

Consider the following C code fragment, which we use to motivate and illustrate
our technique:
\begin{quote}
\begin{verbatim}
f = ...;          (a)  
if (f == NULL) {  (b)
        x = 0;    (c)
        *f;       (d)
}
\end{verbatim}
\end{quote}
Consider the predicate {\tt f == NULL} at line {\tt (b)}.  Clearly
this predicate is highly correlated with failure; in fact, whenever it
is true this program inevitably crashes.  An important observation,
however, is that even a ``smoking gun'' such as {\tt f == NULL} at
line {\tt (b)} cannot be a perfect predictor of failure when there are
multiple bugs in the program---since there are other bugs, the program can fail
even if the predicate is false.  Put another way, we assume
that bugs are independent, and there is no reason to believe that
a predicate that is a good predictor for one bug is at all correlated
with any other bug.

The bug in the code fragment above is \termdef{deterministic} with
respect to {\tt f == NULL}: if {\tt f == NULL} is true at line {\tt
(b)}, the program is guaranteed to eventually fail.  In many cases it
is simply impossible to observe the exact conditions that cause
failure; for example, buffer overrun bugs in a C program may or may
not cause the program to crash depending on runtime system decisions
about how data is laid out in memory.  Such bugs are
\termdef{non-deterministic} with respect to every predicate that we instrument:
even for the best predictor $P$, it is possible that $P$ is true and
still the program terminates normally.  In the example above, if we replace line
{\tt (d)} by
\begin{quote}
\begin{verbatim}
if (random) f = ... some valid pointer ...;
*f;
\end{verbatim}
\end{quote}
then the bug becomes non-deterministic.

To summarize, even for predicates that truly are the causes of bugs, we can neither assume that 
when the predicate is true that
the program fails nor that when the predicate is false that
the program succeeds. But we can express the probability that a predicate
being true implies failure.  Let $\fail$ be an atomic predicate that is
true for failing runs and false for successful runs.  We want to compute:
\[ \crash(P) = \prob(P \Rightarrow \fail) \]
for every predicate $P$ over the set of all runs.  Let $\#S(P)$ be the number
of successful runs in which $P$ is observed to be true, and let $\#F(P)$ be the number of
failing runs in which $P$ is observed to be true.  Then we have
\[ \crash(P) = \frac{\#F(P)}{\#S(P) + \#F(P)} \]

Notice that $\crash(P)$ is not affected by the set of runs in which
$P$ is observed to be false.  Thus, if $P$ is the cause of a bug, the
causes of other independent bugs do not affect $\crash(P)$.  
Also note that runs in which $P$ is not observed at all (either because
the line of code on which $P$ is checked is not reached, or the line is reached
but $P$ is not sampled) have no effect on $\crash(P)$.
Finally, obvserve that the definition of $\crash(P)$
generalizes the idea of deterministic and non-deterministic bugs.  A
bug is deterministic for $P$ if $\crash(P) = 1.0$ or, equivalently,
$P$ is never observed to be true in a successful run ($\#S(P) =
0$) and $P$ is observed to be true in at least one failing run ($\#F(P) > 0$). 
If $\crash(P) < 1.0$ then the bug is non-deterministic, with
lower scores showing weaker correlation between the predicate and
program failure.

As we shall show, $\crash(P)$ is a useful measure, but it is not good
enough for step (1) of our algorithm. To see this, consider again the
code fragment given above (in its original form, not with the
modification to make the bug non-deterministic).  At line {\tt (b)} we
have $\crash(\mbox{\tt f == NULL}) = 1.0$, so this predicate is a good
candidate for the cause of the bug.  
But on line {\tt (c)} we have the surprising fact that $\crash(\mbox{\tt x == 0}) = 1.0$ as well.
To understand why, observe that the predicate $\crash(\mbox{\tt x == 0})$ is always
true when we reach the point immediately after line {\tt (c)} and, in addition, 
only failing runs reach this line.
Thus $\#S(\mbox{\tt x == 0}) = 0$, and, so long as there is at least one run that
reaches line {\tt (c)} at all, the crash value of {\tt x == 0} at line {\tt (c)} is 1.0.

As the predicate {\tt x == 0} at line {\tt (c)} of the example
shows, just because a predicate has a high $\crash(\ldots)$ score does not
mean it is the cause of a bug.  In the case of {\tt x == 0}, the
decision that eventually causes the crash is made earlier, and the
high $\crash(\ldots)$ score of {\tt x == 0} merely reflects the fact that this
predicate is checked on a path where the program is already doomed.

One way to address this difficulty is to score a predicate not by the chance
that it implies failure, but by how much difference it makes that the predicate
is observed to be true versus simply reaching the line where the predicate is checked.
That is, on line {\tt (c)}, the probability of crashing is already 1.0 regardless
of the value of the predicate {\tt x == 0}, and thus the fact that {\tt x == 0} is
true does not increase the probability of failure at all; this coincides with
the intuition that this predicate is irrelevant to the bug.

This leads us to the following definition:
\[ \context(P) = \prob((P \vee \neg P) \Rightarrow \fail) \]
Now, $P \vee \neg P$ is not the set of all runs, because we are not working in a two-valued logic.
In any given run, neither of $P$ or $\neg P$ may be observed (because the site where this predicate is
sampled is not reached), or one may be observed, or both may be observed (because the statement is executed
multiple times and $P$ is sometimes true and sometimes false).  Thus, $\context(P)$ is the probability that
in the set of runs where the value of $P$ is observed at all, the program fails. We can compute $\context(P)$ as follows:
\[ \context(P) = \frac{\#F(P \vee \neg P)}{\#S(P \vee \neg P) + \#F(P \vee \neg P)} \]

The interesting quantity, then, is 
\[ \increase(P) = \crash(P) - \context(P) \]
which can be read as: How much does $P$ being true increase the probability of failure
over simply reaching the line where $P$ is sampled?  For example, for the predicate {\tt x == 0} on line {\tt (c)},
we have 
\[\crash(\mbox{\tt x == 0}) = \context(\mbox{\tt x == 0}) = 1.0 \]
and so $\increase(\mbox{\tt x == 0}) = 0$.
We can now state our algorithm:
\begin{enumerate}
\item Discard any predicate $P$ where $\increase(P) \leq 0$.

\item Sort the remaining predicates lexicographically first by $\crash(P)$ and then by $\context(P)$.
\end{enumerate}

A few comments on this algorithm are in order.  First, pruning
predicates using $\increase(P) \leq 0$ has many desirable
properties.  It is easy to prove that large classes of irrelevant
predicates always have scores $\leq 0$.  For example, any predicate
that is not reached, that is a program invariant, or that is just
control-dependent on a true cause is eliminated by this test.  It is
also worth pointing out that this algorithm tends to localize bugs at
a point where the condition that causes the bug becomes true, rather than at
the crash site.  For example, in the code fragment given above, the bug is
attributed to the success of the conditional branch test {\tt f ==
NULL} on line {\tt (b)} rather than the pointer dereference on line
{\tt (d)}.  Thus, the cause of the bug discovered by the algorithm
points directly to the conditions under which it occurs, rather than
the line on which it occurs (which is usually available anyway in the
stack trace). \Autoref{sec:experiments:results} gives several examples
of this phenomenon while hunting for real bugs.

The purpose of sorting the surviving predicates in step (2) by
$\crash(\ldots)$ is to ensure that the highest confidence predicates (those
most likely to actually cause crashes) are listed first in the final
result.  We need step (2) because while $\increase(\ldots)$ is very good at
eliminating unimportant predicates, it is rather poorer as a measure
of the most important predicates.  Consider line {\tt (a)} in the
example above.  If {\tt (a)} has, say, a 90\% probability of assigning
a {\tt NULL} pointer to {\tt f}, then the value of
$\increase(\mbox{\tt f == NULL})$ will be be high at line {\tt (a)}
and lower, but still positive, at line {\tt (b)}.  While both {\tt
(a)} and {\tt (b)} contribute to the bug, it is our experience that,
once irrelevant predicates are discarded, it is most natural to
associate a bug with the predicate that gives the highest absolute
chance of failure; these are most likely to be the true causes.  For
instance, in our example, something is certainly wrong by the time we
reach line {\tt (b)}, where taking the true branch leads to a certain
crash, while it is entirely possible that line {\tt (a)} is correct.
Most likely the bug is a typographical error: the programmer meant to
write {\tt f != NULL} as the predicate of the conditional instead of
{\tt f == NULL}.

Finally, it should be clear that the algorithm is efficient.
Step (1) requires only a single pass over the data, and step (2) is
simply sorting.


% LocalWords:  pre


\placeholder{Page budget: 2.}


\section{\placeholder{Experiments}}
\label{sec:experiments}

\subsection{\placeholder{Setup}}
\label{sec:experiments:setup}

\placeholder{Page budget: 1.}

This section discusses background for the experimental results
reported in \Autoref{sec:experiments:results}.  While all software
experiments are difficult to do well, we have learned the hard way
that there are some particular problems that must be addressed to do
our experiment well.  Thus, this section discusses the set-up for our
experiment in some detail, especially how we have compensated for
potential sources of bias. 

The basic framework of our experiment is a straightforward five step
process: 
\begin{enumerate}
\item select an existing software application, 
\item modify the source code to inject bugs into the program, 
\item instrument the modified program,
\item gather results from a large number of runs performed with automatically generated data and 
\item apply our algorithm to the results.  
\end{enumerate}
We discuss each step; the reader who is not interested in
these details may wish to proceed to \Autoref{sec:experiments:results} and
use this section only for reference.

We chose \moss\ \cite{Schleimer:2003:WLA} as our benchmark program.  \moss\ is a
software plagiarism detection service\footnote{That is,
\moss\ detects copying in large sets of programs.  The typical \moss
user is a professor or teaching assistant in a programming course.}
that has been available since the late '90's and has several thousand
users worldwide.  As such, \moss\ has many of the characteristics of
real software: it has users who depend on it, it is constantly
undergoing revision as its purpose and the environment in which it
runs evolves, and it is complex enough to be composed of several
interacting subsystems.  The last point, in particular, means that it
is reasonable that \moss could have multiple bugs simultaneously.
From our point of view, \moss\ has the additional advantage that it
was written and is maintained by one of the authors.

The next step, injecting bugs into the software, is problematic, as
the choice of bugs to include or exclude can dramatically affect the
results.  In our case, we sought to use ``real'' bugs as much as
possible, while also seeking to have some variety among the bugs we
included.  Nearly all of the bugs were taken directly from the bug
logs for \moss.  In some cases the code had evolved since the original
bug was fixed, in which case we had to judge how to modify the
bug to inject it into the code.  We feel the modifications, where
needed at all, were straightforward.  We also included three bugs that
were not \moss\ bugs.  One of these is a known bug from another system
where there is an obviously analogous place to add that bug to \moss\
(see below). The other two are duplicates of two different buffer
overrun bugs in \moss.  In each case, we restored the original bug,
and then added a second, very similar buffer overrun in a different
place, the purpose being to see if our algorithm could not only detect
the overruns, but also distinguish between them.

We briefly describe the nine bugs we added to \moss:
\begin{enumerate}
\item To correctly report the location of duplicate code \moss\ must
track line numbers.  We introduced a bug that causes the number of
lines in C-style multi-line comments to be counted incorrectly.  The
bug only occurs under a special set of circumstances: the option to
match comments must be on (normally \moss\ ignores comments
completely, and that is a separate code path with no bug), the
programs involved must have C multi-line comments, and in addition the
position of these comments must ultimately affect the position of
reported matches.  Note that this bug is not only non-deterministic in
the sense defined in \Autoref{sec:algorithm}, it also does not
cause the program to crash; the programs simply generates incorrect
output.

\item \moss\ has the option to dump its internal data structures in a
binary file format that can be reloaded quickly; these external files
are called {\em databases}.  We modified the code to remove the check for a
null {\tt FILE} pointer in the case that the database cannot be opened
for writing.  This bug is analogous to one reported in {\tt ccrypt}
\cite{Selinger:2003:cqual}.  This is a deterministic bug, and in fact the
program crashes almost immediately after failing to open the file.

\item Loading a \moss\ database is fairly complex, as a number of data
structures must be kept in sync.  We removed an array bounds update
in the database loading routine, so that even though a database was
loaded, the pointer to the end of one array {\tt A} was not moved to
reflect that new data had been added to the end of {\tt A}. The
program behaves normally unless a second database is loaded, at which
point the second database at least partially overwrites that portion
of the first database stored in {\tt A}.  This bug has unpredictable
effects.  Depending on what files are compared and the contents of the
databases loaded, the result might be that the program terminates with
correct output, that it terminates with incorrect output, or that it
crashes.  This was a particularly difficult bug to find originally.

\item We removed a size check that prevented users from supplying command-line arguments
that could cause the program to overrun the bounds of an array.  When
this bug is triggered the program may terminate with correct output,
terminate with incorrect output, or crash.


\item \moss\ handles Lisp programs differently from all other languages;
at one time all languages where handled in the same manner, but the
others have been gradually ported to an improved algorithm.  The Lisp
processing involves a standard hash table; we removed one of the
end-of-bucket checks, which causes a crash when the program scans to
the end of a hash bucket and tries to dereference a \texttt{NULL} pointer.  This
bug only occurs for Lisp programs, but does reliably crash the program
when it is touched.

\item For efficiency \moss\ preallocates a large area of memory for its primary data structure.
When this area of memory is filled, the program should fail
gracefully.  We removed the out-of-memory check.  The original bug
was a bit more complex, but cannot be reproduced exactly because this
portion of the code has been substantially revised.  

\item \moss\ has a routine that scans an array for multiple copies of the same passage of code within
a single file; duplicate code within a file is treated differently than duplicate code across files.
We removed the limit check that prevents the code from searching past the end of the array.  This is another
buffer overrun, but of a different kind.  First, whether the overrun occurs is very data dependent and in fact it is
difficult to construct a test case by hand that triggers the bug.  Second, the routine in question only reads
past the end of the array (no memory locations are written), so it is quite likely that the program will
succeed in spite of the error.  This bug is synthetic (it never occurred in \moss) but is derived from bug \#8.

\item This is a variant on bug \#7, in another routine that deals with duplicates.
Again, the bug allows the program to read past the end of an array
while searching for particular data values.  In contrast to bug \#7,
bug \#8 occurs under an even rarer set of circumstances.  In
fact, this bug was never known to have caused a failure in \moss; it
was discovered by a code review.

\item This bug is a variant of bug \#4, but involves a different command-line argument and
a different array.
\end{enumerate}

In summary, the nine bugs are all either real bugs in \moss\ or bugs
closely related to real bugs in \moss\ or other programs.  The bugs
range from typical C coding errors (e.g., \texttt{NULL} pointer dereferences
and array overruns) to high-level violations of a system's internal
invariants (e.g., bugs \#1 and \#3).

To allow us to measure the accuracy of our techniques we also added code to \moss\ to 
log when each bug was triggered.  For example, for bug \#1 we added:
\begin{verbatim}
void check_bug_1(const char *yytext, int yyleng)
{
  static int seen;

  if (memchr(yytext, '\n', yyleng))
    reportOnce(&seen, 1, "... log message ...");
}
\end{verbatim}
The purpose of the static variable {\tt seen} is to ensure that each bug is logged only the first time
it occurs in a run (some of the bugs can occur multiple times during one execution).  

In the next step we ran our source-to-source instrumentor on \moss,
modified to include the bugs and the logging code.  Recall that our
system correlates predicates derived from the program source with the
success or failure of an execution to isolate bugs.  Given the log
function for bug \#1 above, it is very likely that our system would
report that whenever the predicate of the conditional {\tt
memchr(\ldots)} is true, the program is likely to fail.  To prevent
our logging code from ``giving away'' the causes of the bugs to our
bug isolation algorithm, the logging functions were isolated in a
separate file that was not processed by our instrumentor.  We also
excluded from instrumentation code automatically generated by the tool
{\tt flex}.  While examining predicates on {\tt flex}'s internal state
would quite possibly yield some useful clues about the sources of
bugs, it is very unlikely that any programmer working on \moss\ would
be able to interpret such predicates, and so it is better to exclude them from
consideration.

After instrumenting \moss\ we ran both the buggy version and the
original version on the same random inputs. We recorded whether the buggy
version succeeded or failed by examining its exit code and, in the
case that it terminated normally, by checking whether the output of
both versions matched.  In practice, we envision that users would have
a way to give simple feedback on program executions in the case that a
program terminates normally but produces incorrect output.  We have
modeled that in our experiment by recording one bit (success or
failure) for each run based on whether its output matches that of the
unmodified \moss\ on the same inputs.

The randomly generated inputs are produced according to the following
scheme.  A different probability distribution is associated with each
command line option.  For the numeric options, the distribution gives
the probability that the option takes on a small, medium, or large
value, or is absent altogether.  For options that are essentially
enumerated types (such as the programming language used) the
distribution just includes the probability of each element of the
enumeration as well as the probability that the option is absent.
Depending on the programming language chosen, files to submit to
\moss\ are chosen randomly from a collection of thousands of C, Lisp,
and Java files.  The number of programs to submit with each run is
another probability distribution, which guarantees that there are at
least some runs with a very small number of files and some with a very
large number of files submitted.  

In the process of performing the experiment we discovered several ways
in which our inputs were not as random as they could be---we had
accidentally coupled two or more choices that could be independent.
We removed removed many such sources of coupling among input parameters that we
discovered.  Note that having more variation in the inputs helps our
techniques, as increased variation reduces the chances that a program
predicate will coincidentally appear to be correlated with failure.

Finally, the infrastructure for this experiment was sufficiently complex
that we found it necessary to automatically discard random executions that
failed in ways we could not handle.  For example, rarely the buggy version of
\moss\ would hang instead of crashing, in which case it was eventually killed
by a watchdog process we set up for that purpose.  In these cases no report
was generated, so we could not make use of the run.

The analysis of the results of applying our algorithm is the
subject of \Autoref{sec:experiments:results}.

%% LocalWords:  ccrypt


\subsection{\placeholder{Results}}
\label{sec:experiments:results}

\placeholder{Page budget: 2, including pictures.}


\section{\placeholder{Future Work / Alternate Approaches}}
\label{sec:future-work}

\placeholder{Page budget: \nicefrac{1}{2}.}

Our experience with \moss suggests several key areas for future
development.

In the experiments reported here, sampling is dynamically uniform:
each crossing of each instrumentation site has the same fixed chance
of being sampled or skipped.  However not all code is equally
interesting.  The details of a one-time configuration action may have
more impact on a bug than some bulk initialization loop that iterates
millions of times.  Two generalizations of our approach support
non-uniform sampling.

One might replace the single global ``next sample'' countdown with
several countdowns having a variety of means.  Each acyclic region
uses one of these counters for all of its sites.  Thus,
heavily-trafficked code might be sampled at a rate of
\nicefrac{1}{1000}, while moderate-use code uses \nicefrac{1}{100} and
rarely executed initialization and error-handling code uses
\nicefrac{1}{10} or even \nicefrac{1}{1} (i.e.\ no sampling; complete
data collection).

A complementary extension is to allow individual sites to decrement
the ``next sample'' countdown more than once.  Decrementing once is
equivalent to tossing a biased coin once and only taking a sample if
that toss comes up (e.g.) heads.  Decrementing twice simulates taking
a sample if either of two tosses comes up heads.  For a basic
underlying countdown with mean $d$, decrementing $n$ times yields an
effective sampling rate of $(1 - (1 - \frac{1}{d})^n)$.  For purposes
of checking countdown thresholds at the tops of acyclic regions, a
site that decrements $n$ times is equivalent to $n$ sites that
decrement once.  Thus each global ``next sample'' countdown actually
supports a family of sampling rates selectable on a per-site basis.

\placeholder{The above assumes that we have already given a refresher
  on global countdowns and threshold checks earlier in the paper.
  Rewording will be needed if we don't get into that amount of
  detail.}

\placeholder{Yet to be addressed: how you actually decide what rate to
  use for a given site.  Presumably some mix of ad-hoc guessing,
  coverage information from in-house testing, and dynamic adjustment
  in the field based on what you're seeing in the feedback reports
  you've gotten so far.}

\placeholder{Yet to be addressed: need for better statistical models.
  This is a bit vague, but should at least discuss problems we
  encountered using logistic regression in the Moss experiment.}

The sampling instrumentor uses several program analyses to boost the
performance of instrumented code.  However, the post-run data analysis
described in \Autoref{sec:algorithm} is largely ignorant of program
structure.  Exploiting properties visible in the source code can allow
further filtering of suspect predicates.  We expect that in any given
failed execution, there are two points of greatest interest to an
engineer: the place where things first started to go wrong, and the
place where an immediate failure became truly inevitable.  Thus,
predicates with high crash scores \placeholder{Is ``crash score'' the
  right term here?  Check against \Autoref{sec:algorithm}.} are most
interesting when they appear close to program entry or close to the
location of an actual crash.

This suggests an approach based on traversal of the control flow
graph.  To find earliest causes, search forward from program entry for
points at which the chance of failure substantially increases.  To
find latest causes, search backward from crash locations for points at
which the chance of failure substantially decreases.  Predicate
observations provide additional clues, and could be incorporated to
further restrict the search to only those paths which are consistent
with observed behavior.  \placeholder{The preceding feels too detailed
  for an algorithm we haven't actually implemented.  I'd rather
  provide sketches of more distinct ideas rather than presenting just
  one idea in this much detail.}

\section{Related Work}
\label{sec:related-work}

\placeholder{Page budget: 1.}

The Daikon project \cite{ernst2001} monitors instrumented applications
to discover likely program invariants.  It collects extensive trace
information at run time and uses this offline to accept or reject any
of a wide variety of guessed candidate predicates.  The DIDUCE project
\cite{ICSE02*291} tests a more restricted set of predicates within the
client program, and attempts to relate state changes in candidate
predicates to manifestation of bugs.  Both projects assume complete
monitoring, such as within a controlled test environment.  Our goal is
to use lightweight partial monitoring, suitable for deployment to end
users.  We never have complete information, and therefore must use a
more probabilistic approach: we wish to identify broad trends over
time that correlate predicate violations with increased likelihood of
failure.

\termdef{Software tomography} as realized through the GAMMA system
\cite{PASTE'02*2,Orso:2003:LFDIART} shares our goal of low-overhead
distributed monitoring of deployed code.  GAMMA collects code coverage
data to support a variety of code evolution tasks.  Our
instrumentation exposes a broader family of data- and
control-dependent predicates on program behavior and uses randomized
sparse sampling to control overhead.  We observe, however, that the
predicates injected by our instrumentor can approximate coverage: over
many runs, the sum of all predicate counters at a site converges on
the relative coverage of that site.

Efforts to directly apply statistical modeling principles to debugging
have met with mixed results.  Early work in this area by Burnell and
Horvitz \cite{Burnell:1995:SCM} uses a mix of program slicing and
Bayesian belief networks to filter and rank the possible causes for a
given bug.  Empirical evaluation show that the slicing component alone
finds 65\% of bug causes, while the probabilistic model correctly
identifies another 10\%.  This additional payoff may seem small,
especially in light of the effort, measured in multiple man-years,
required to distill experts' often tacit knowledge into a formal
belief network.  However, the approach does illustrate one strategy
for integrating information about program structure into the
statistical modeling process.

In more recent work, Podgurski et al.\ \cite{ICSE`03*465} apply
statistical feature selection, clustering, and multivariate
visualization techniques to the task of classifying software failure
reports.  The intent is to bucket each report into an equivalence
group believed to share the same underlying cause.  Features are
derived offline from fine-grained execution traces without sampling;
this reduces the noise level of the data but may be impractical to
deploy outside of a controlled testing environment.  As in our own
earlier work, Podgurski uses logistic regression to select features
which are highly predictive of failure.  \placeholder{Is it worth
  noting that we use different strategies for limiting the size of the
  set of selected features?  We use regularized logistic regression
  whereas Podgurski applies standard logistic regression to randomly
  selected subsets of the complete feature set and keeps the
  best-performing subset.}  Clustering tends to identify small, tight
groups of runs which do share a single cause but which are not always
maximal.  (That is, one cause may be split across several clusters.)

Studies that attempt real-world deployment of monitored software must
address a host of practical engineering concerns, from distribution to
installation to user support to data collection and warehousing.
Elbaum and Hardojo \cite{Elbaum:2003:DISATA} have reported on a
limited deployment of instrumented Pine binaries.  Their experiences
have helped to guide our own design of a wide public deployment of
applications with sampled instrumentation, presently underway
\cite{Liblit:2003:CBIP}.

\placeholder{More recent work of interest:

  \begin{itemize}
  \item S.\ Elbaum, S.\ Kanduri and A.\ Andrews, ``Anomalies as
    Precursors of Field Failures, International Symposium of Software
    Reliability Engineering'', IEEE, November 2003.
    
  \item S.\ Elbaum and M.\ Hardojo, ``An Empirical Study on Profiling
    Strategies for Released Software and Their Application to QA
    Activities'', Technical Report TR03-09-01, University of Nebraska
    - Lincoln, September 2003.
  \end{itemize}
  
  Cannot find either online, but have written to Elbaum asking him for
  copies.}

For some highly available systems, even a single failure must be
avoided.  Once the behaviors that predict imminent failure are known,
automatic corrective measures may be able to prevent the failure from
occurring at all.  The Software Dependability Framework (SDF)
\cite{Gross:2003:PSMUST} uses the multivariate state estimation
technique to model and thereby predict impending system failures.
Instrumentation is assumed to be complete and is typically
domain-specific, whereas our sampled predicates cast a wider, less
specialized net.  \placeholder{We understand through informal
  communication that the SDF is able to anticipate when a player is
  about to lose an instrumented game of Tetris, and can intervene by
  removing rows to allow the game to continue.  But maybe I shouldn't
  say that, as it kind of makes their system sound like a joke.}

\placeholder{Cannot find any more recent work by these people in this
  area.  Where did they all go?  Porter has plenty of other recent
  work, but apparently nothing related.  Gross and McMaster have zero
  publication information on their home pages, while Umranov and Votta
  seem to have vanished entirely.  Have written to Gross and Porter
  asking if they have anything more recent I should look at.}

\section{Conclusions}
\label{sec:conclusions}

\bibliography{cacm1990,icse02,icse03,misc,paste02,pldi03,pods,ramss,refs}
\placeholder{Page budget for references: 1.}

\end{document}

%% LocalWords:  DIDUCE Burnell Horvitz Podgurski Elbaum Hardojo SDF
%% LocalWords:  cacm icse ramss
