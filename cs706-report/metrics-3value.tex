
\documentclass{article}
\usepackage{graphicx}
\begin{document}

\title{CBI }
\maketitle
\begin{abstract}



\end{abstract}


\section{Introduction}

\subsection{Three valued logic}

Three-valued logic is applicable in the situation where some
predicate value is uncertain (i.e. neither true or false). Under
CBI's sampling framework, the two branches of a complex predicate
may not be sampled at one program run in general. If we model the
value of a un-sampled predicate as unknown, the three-value logic
truth tables (Table \ref{and} and Table \ref{or}) are readily
applicable for both conjunctions and disjunctions. The truth value
tables are easy to understand: when we evaluate a conjunction and
find one of its branch is $false$, the final value must be $false$
regardless the value of the other branch. A similar short-circuit
evaluation can be used for disjunctions when we find one of each
branches is true.


\begin{table}[h]
  \centering
  \begin{tabular}{|c|ccc|}
  \hline
  % after \\: \hline or \cline{col1-col2} \cline{col3-col4} ...
  $P_1$$\backslash$$P_2$ & T & F & $?$ \\
  \hline
  T & T & F & ? \\
  F & F & F & F \\
  ? & ? & F & ? \\
  \hline
\end{tabular}
  \caption{3-valued Truth Table for $P_1$$\wedge$$P_2$}\label{and}
\end{table}


\begin{table}[h]
  \centering
  \begin{tabular}{|c|ccc|}
  \hline
  % after \\: \hline or \cline{col1-col2} \cline{col3-col4} ...
  $P_1$$\backslash$$P_2$ & T & F & $?$ \\
  \hline
  T & T & T & T \\
  F & T & F & ? \\
  ? & T & ? & ? \\
  \hline
\end{tabular}
  \caption{3-valued Truth Table for $P_1$$\vee$$P_2$}\label{or}
\end{table}





\subsection{Increase}

\begin{equation}\label{increase}
    Increase(P)' = \frac{F(P)'}{S(P)'+F(P)'} - \frac{F(P\;\;Observed)'}{S(P\;\;Observed)'+F(P\;\;Observed)'}
\end{equation}

where for a conjunction, we have

%\begin{equation}\label{conjunction}
%    F(P)' = Min(F(P_1),F(P_2)) \\
%    S(P)' = 0
%\end{equation}

\begin{eqnarray}
 \nonumber %to remove numbering (before each equation)
  F(P)' &=&  Min(F(P_1),F(P_2)) \\
 \nonumber
  S(P)' &=& 0 \\
 \nonumber
  F(P\;Observed)' &=& F(P)' \\
 \nonumber
  S(P\;Obs)' &=&
  Max(S(P_1\;Obs)+S(P_2\;Obs)-S(P_2),S(P_2\;Obs)+S(P_1\;Obs)-S(P_1))
\end{eqnarray}

and for a disjunction,

\begin{eqnarray}
 \nonumber %to remove numbering (before each equation)
  F(P)' &=&  F(P_1) + F(P_2) \\
 \nonumber
  S(P)' &=&  Max(S(P_1),S(P_2)) \\
 \nonumber
  F(P\;Observed)' &=& 0 \\
 \nonumber
  S(P\;Obs) &=& 1
\end{eqnarray}



\section{Measuring Usefulness of Complex Predicates}

While complex predicates could help to better predict bugs, in the
worst case, we will see quadratic increase in the number of
possible combinations of two single predicates. In our
experiments, we often observe hundreds of complex predicates with
similar or even identical high scores. The overwhelming number of
predicates makes it extremely laborious for users to find the true
problem cause and then debug. In this section, we several metrics
to reduce the number of complex predicates.

Our basic idea is to model programmer debugging efforts as a
filtering metric. We assume that when a CBI user tries to find the
cause of a bug given a complex predicate, he or she has to browser
through (e.g. doing a BFS like search) all the statements executed
between the two branches. Therefore, the further apart the two
branches is, the more efforts a user has to devote. Suppose the
two predicates of a complex predicate are associated with two
nodes in a Control Flow Graph (CFG). We define the distance
between two branches of a complex as the graph distance of the two
nodes in the CFG.

Another idea to break the ties of complex predicates is to
consider the correlation between their branches. Intuitively a
conjunction (or a disjunction) with two relative independent
branches is less interesting because it doesn't help users much to
find anything extra besides the two individual predicates. Here we
define the correlation of two predicates based on the notion of
program dependency graph. Given a single predicate $P$, we define
its $predecessor set$ as the set of vertices in the CFG which can
influence the outcome of $P$. The correlation between two
predicates of a complex predicate is thus defined as the number of
vertices in the intersection of the two predecessor sets. A
complex predicate with a higher correlation will be ranker higher.

We use CodeSurfer to build the program CFG and dependency graph.
CodeSurfer APIs are also used in computing the vertex distance and
intersections of predecessor sets.

The above two metrics could be applied both $proactively$ and
$reactively$. A proactive use of the metrics will prune away
complex predicates whose metric values fall below a certain
threshold and thus eliminate them from consideration in forming
larger complex predicates. A reactive use of the metrics will
retain all the predicates but break ties by giving higher ranks to
those with higher metric values. This is desirable if neither
computing time or space is a concern.







\end{document}
